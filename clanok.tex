% Metódy inžinierskej práce

\documentclass[10pt,twoside,slovak,a4paper]{article}

\usepackage[slovak]{babel}
%\usepackage[T1]{fontenc}
\usepackage[IL2]{fontenc} % lepšia sadzba písmena Ľ než v T1
\usepackage[utf8]{inputenc}
\usepackage{graphicx}
\usepackage{url} % príkaz \url na formátovanie URL
\usepackage{hyperref} % odkazy v texte budú aktívne (pri niektorých triedach dokumentov spôsobuje posun textu)

\usepackage{cite}
%\usepackage{times}

\pagestyle{headings}

\title{Vyhľadávanie pomocu AI a vpliv AI na lepšie vyhladávanie. \thanks{Semestrálny projekt v predmete Metódy inžinierskej práce, ak. rok 2023/24, vedenie: MSc.Mirwais Ahmadzai}} % meno a priezvisko vyučujúceho na cvičeniach

\author{Richard Gajarský\\[2pt]
	{\small Slovenská technická univerzita v Bratislave}\\
	{\small Fakulta informatiky a informačných technológií}\\
	{\small \texttt{...@stuba.sk}}
	}

\date{\small 30. september 2023} % upravte



\begin{document}

\maketitle

\begin{abstract}
\ldots 
Využitie umelej inteligencie (AI) v oblasti vyhľadávania má revolučný vplyv na spôsob, akým ľudia nájdu informácie online. Tento abstrakt sa zameriava na významný vplyv AI na vylepšenie vyhľadávacích procesov. AI algoritmy, ako je strojové učenie a hlboké učenie, umožňujú vyhľadávacím systémom porozumieť kontextu a osobným preferenciám používateľov. To vedie k presnejším a relevantným výsledkom vyhľadávania.

Ďalší aspekt je rozvoj chatbotov a hlasových asistentov, ktorí dokážu komunikovať s používateľmi a pomáhať im pri vyhľadávaní informácií. AI tiež umožňuje analyzovať obrovské množstvo dát rýchlejšie a efektívnejšie, čo zlepšuje vyhľadávanie informácií v medicíne, vede, obchode a mnohých ďalších odvetviach.

Napokon, AI zvyšuje aj bezpečnosť vyhľadávacích systémov tým, že odhaľuje spam, phishingové útoky a dezinformácie. Celkovo gľadávanie pomocou AI výrazne zjednodušuje proces nájdenia informácií a prináša používateľom presné a relevantné výsledky v reálnom čase.

\end{abstract}



\section{Úvod}

V súčasnom digitálnom veku, kedy je prístup k informáciám rýchly a jednoduchý, má vyhľadávanie na internete zásadný význam pre každodenný život. S nástupom umelej inteligencie (AI) do oblasti vyhľadávania sme svedkami dramatických zmien v spôsobe, ako ľudia získavajú potrebné informácie. AI algoritmy, ako strojové učenie a hlboké učenie, transformujú vyhľadávacie systémy a umožňujú im lepšie porozumieť potrebám a preferenciám používateľov. Tento článok sa zameria na vplyv AI na vyhľadávanie a jeho dôsledky na presnosť, efektívnosť a bezpečnosť tohto procesu. Naša analýza je založená na výskumoch a literatúre z oblasti umelej inteligencie a informačných vied.



\section{Vývoj AI v kontexte vyhľadávania}
\label{sec:vývoj-ai}

V tejto sekcii sa budeme podrobnejšie venovať vývoju umelej inteligencie (AI) v kontexte vyhľadávania na internete. Predstavíme kľúčové technológie, ktoré formujú moderné vyhľadávacie systémy a umožňujú im lepšie porozumieť potrebám používateľov.

\subsection{Strojové učenie a jeho úloha v AI}
\label{subsec:strojové-učenie}

Strojové učenie hrá kľúčovú rolu v rozvoji umelej inteligencie a vyhľadávania. Algoritmy strojového učenia, ako je rozhodovací strom a metóda podporných vektorov, sa stali základnými nástrojmi pre vylepšenie vyhľadávania (Sebastiani, 2002). Tieto algoritmy umožňujú vyhľadávacím systémom naučiť sa z dát a zlepšiť presnosť výsledkov (Mitra, Singhal, & Buckley, 1998). Ďalším prínosom strojového učenia je jeho schopnosť poskytovať personalizované odporúčania pre používateľov (Chen, Zhai, & Lafferty, 2006).

\subsection{Hlboké učenie v kontexte vyhľadávania}
\label{subsec:hlboké-učenie}

Hlboké učenie, konkrétne konvolučné neurónové siete (CNN) a rekurentné neurónové siete (RNN), sa stalo kľúčovým nástrojom pre spracovanie komplexných úloh v kontexte vyhľadávania (Gao et al., 2019). Táto technológia umožňuje vyhľadávacím systémom efektívnejšie spracovávať textový a obrazový obsah. Výsledkom je vylepšená schopnosť systémov porozumieť obsahu a kontextu používateľských dopytov.

\subsection{Chatboti a hlasoví asistenti ako súčasť AI v vyhľadávaní}
\label{subsec:chatboti-hlasoví-asistenti}

Komunikácia s vyhľadávacími systémami sa stáva stále interaktívnejšou vďaka chatbotom a hlasovým asistentom (Cambria, White, 2014). Tieto technológie umožňujú používateľom komunikovať s vyhľadávacími systémami prirodzenejším spôsobom. Chatboti a hlasoví asistenti môžu pomôcť používateľom pri vyhľadávaní informácií a poskytovať rýchle a personalizované odpovede.

\section{Vplyv AI na presnosť vyhľadávania}
\label{sec:vplyv-ai-presnost}

Táto sekcia sa zameriava na vplyv umelej inteligencie (AI) na presnosť vyhľadávania a poskytuje detailnejší pohľad na rôzne aspekty tejto problematiky.

\subsection{Personalizované vyhľadávanie a odporúčania}
\label{subsec:personalizovane-vyhledavanie}

Jedným z významných prínosov AI v oblasti vyhľadávania je schopnosť personalizovať výsledky pre každého používateľa. AI algoritmy sledujú používateľské správanie a preference na základe histórie vyhľadávania a interakcií. Týmto spôsobom môžu poskytovať odporúčania, ktoré sú relevantné pre konkrétneho používateľa (Lops, De Gemmis, & Semeraro, 2011).

\subsection{Efektívnosť vyhľadávania v špecifických oblastiach}
\label{subsec:efektivnost-vyhledavani}

AI zohráva dôležitú úlohu pri zvyšovaní efektívnosti vyhľadávania v špecifických oblastiach. V rámci medicíny, vedy a iných odvetví pomáha AI identifikovať a vyhodnocovať relevantné informácie rýchlejšie a presnejšie. Tým prispieva k zlepšeniu výskumných a diagnostických procesov.

\section{AI a rýchlejšie analyzovanie dát}

\subsection{Aplikácie v medicíne}

\subsection{Vplyv na vedu a výskum}

\section{Bezpečnosť vyhľadávania s využitím AI}

\subsection{Detekcia spamu a phishingu}

\subsection{Boj proti dezinformáciám}














\section{Záver} \label{zaver} % prípadne iný variant názvu



%\acknowledgement{Ak niekomu chcete poďakovať\ldots}


% týmto sa generuje zoznam literatúry z obsahu súboru literatura.bib podľa toho, na čo sa v článku odkazujete
\bibliography{literatura}
\bibliographystyle{plain} % prípadne alpha, abbrv alebo hociktorý iný
\end{document}
